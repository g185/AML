%%%%%%%%%%%%%%%%%%%%%%%%%%%%%%%%%%%%%%%%%
% fphw Assignment
% LaTeX Template
% Version 1.0 (27/04/2019)
%
% This template originates from:
% https://www.LaTeXTemplates.com
%
% Authors:
% Class by Felipe Portales-Oliva (f.portales.oliva@gmail.com) with template 
% content and modifications by Vel (vel@LaTeXTemplates.com)
%
% Template (this file) License:
% CC BY-NC-SA 3.0 (http://creativecommons.org/licenses/by-nc-sa/3.0/)
%
%%%%%%%%%%%%%%%%%%%%%%%%%%%%%%%%%%%%%%%%%

%----------------------------------------------------------------------------------------
%	PACKAGES AND OTHER DOCUMENT CONFIGURATIONS
%----------------------------------------------------------------------------------------

\documentclass[
	12pt, % Default font size, values between 10pt-12pt are allowed
	%letterpaper, % Uncomment for US letter paper size
	%spanish, % Uncomment for Spanish
]{fphw}

% Template-specific packages
\usepackage[utf8]{inputenc} % Required for inputting international characters
\usepackage[T1]{fontenc} % Output font encoding for international characters
\usepackage{multicol, latexsym, amsmath, amssymb}
\usepackage{blindtext}
\usepackage{subcaption}
\usepackage{caption}
\usepackage{wrapfig}
\usepackage{tabu}
\usepackage[dvipsnames]{xcolor}
\usepackage{floatflt}

\usepackage{graphicx} % Required for including images

\usepackage{booktabs} % Required for better horizontal rules in tables

\usepackage{listings} % Required for insertion of code

\usepackage{enumerate} % To modify the enumerate environment


%----------------------------------------------------------------------------------------
%	ASSIGNMENT INFORMATION
%----------------------------------------------------------------------------------------

\title{Assignment 1} % Assignment title

\author{Giuliano Martinelli 1915652, Gabriele Giannotta 1909375, Mario Dhimitri 1910181 } % Student name

\date{October 30th, 2020} % Due date

\institute{Sapienza Università di Roma \\ Data Science} % Institute 

\class{Advanced Machine Learning} % Course or class name

\professor{Fabio Galasso} % Professor 

%----------------------------------------------------------------------------------------

\begin{document}

\maketitle % Output the assignment title, created automatically using the information in the custom commands above

%----------------------------------------------------------------------------------------
%	ASSIGNMENT CONTENT
%----------------------------------------------------------------------------------------

\section*{Report 1 - Image Filtering}
\section { Question 1D}

In this first report, our purpose is to check the effects of convolution with different pairs of kernels. To start, we consider a sample image in which only the central pixel has a non-zero value.

\begin{figure*}[h!]
    \centering
    \includegraphics[height=2in]{img/1d/sample1.png}
     \caption{Sample Image}
\end{figure*}

Then, applying different filter combinations, we obtained the following results:

\begin{figure*}[h!]
    \centering
    \includegraphics[height=2in]{img/1d/fig2.jpg}
     \caption{Sample Image after filter combinations}
\end{figure*}


\begin{enumerate} 
 \item First Gx, then $\mathrm{Gx^{T}}$
		 \\ Applying this filter combination we see normal result. In this case we can see how the gaussian distribution is respected in relation with the filtering of a point in the center of the image. In fact the distribution is more dense in the center and then is less dense when we go far away.
 \item First Gx, then $\mathrm{Dx^{T}}$
		\\Now we see another important result. applied to the original point are two filters. One is the gaussian filter the other is the gaussian derivative filter. 
 \item First $\mathrm{Dx^{T}}$, then Gx
		\\ Result is the same, commutative property.
 \item First Dx, then $\mathrm{Dx^{T}}$
		\\ Now is a derivative with a derivative vertical wave and then horizontal wave. So, it's like having two waves.
 \item First Dx, then $\mathrm{Gx^{T}}$
		\\ In this case is not transposed, for this reason it's not vertical but horizontal.
 \item First $\mathrm{Gx^{T}}$, then Dx
		\\ Same as point  5.
 \end{enumerate}


\section { Question 1E}

Given the two original images graf.png and gantrycrane.png:

\begin{figure*}[h!]
    \centering
    \begin{subfigure}[t]{0.5\textwidth}
        \centering
        \includegraphics[height=1.4in]{img/1e/graf.png}

    \end{subfigure}%
    ~ 
    \begin{subfigure}[t]{0.4\textwidth}
        \centering
        \includegraphics[height=1.4in]{img/1e/gantrycrane.png}

	\end{subfigure}
	\caption{Original Images graf.png and gantrycrane.png}
\end{figure*}


After converting them in greyscale, from the function gaussderiv we applied the filter combination of Gaussian smoothing vertically and derivative of Gaussian horizontally \(Fig. 4\) and the combination of Gaussian smoothing horizontally and derivative of Gaussian vertically \(Fig. 5\):


\begin{figure*}[h!]
    \centering
    \begin{subfigure}[t]{0.5\textwidth}
        \centering
        \includegraphics[height=2in]{img/1e/graf1.png}

    \end{subfigure}%
    ~ 
    \begin{subfigure}[t]{0.4\textwidth}
        \centering
        \includegraphics[height=2in]{img/1e/gantry1.png}

	\end{subfigure}
	\caption{Horizontal smoothing and vertical derivative}
\end{figure*}



\begin{figure*}[h!]
    \centering
    \begin{subfigure}[t]{0.5\textwidth}
        \centering
        \includegraphics[height=2in]{img/1e/graf2.png}

    \end{subfigure}%
    ~ 
    \begin{subfigure}[t]{0.4\textwidth}
        \centering
        \includegraphics[height=2in]{img/1e/gantry2.png}

	\end{subfigure}
	\caption{Vertical smoothing and horizontal derivative}
\end{figure*}


In Fig.4 it's easy to notice that the horizontal edges are highlighted and the vertical ones are ignored. While in Fig. 5 we can see the opposite: vertical edges are highlighted and the horizontal ones ignored. \\
The main reson for using smoothing before applying derivative filter is to reduce the noise and 





\newpage
\section*{Report 2 - Object Identification}


In order to find the best combination to get a better result, we computed the recognition rate for all the possible combinations of the three type of distance (intersect, l2, chi2) with respect to the histogram functions (rgb, rg, dxdy), considering 6 different number of bins (5,10,15,20,30,50) for each combination. After that, the obtained results were inserted in a dataframe and analyzed with Pandas tools. \\
From the 54 combinations analyzed, the following results were obtained:\\ \\


\begin{figure*}[h!]
    \centering
    \begin{subfigure}[t]{0.5\textwidth}
        \centering
        \includegraphics[height=0.9in]{img/best_combination.png}
         \caption{Best Combination}
    \end{subfigure}%
    ~ 
    \begin{subfigure}[t]{0.5\textwidth}
        \centering
        \includegraphics[height=0.9in]{img/worst_combination.png}
         \caption{Worst Combination}
	\end{subfigure}
	\caption{Best and Worst Combination}
\end{figure*}


The best combination found is: \{Histogram: rgb; Distance: Intersect; Number of Bins: 15\}, with a number of matches of 81 out of 89 ( Recognition Rate = 0.91 ).\\ 

The worst combination found is: \{Histogram: rgb; Distance: chi2; Number of Bins: 50\}, with a number of matches of 29 out of 89 ( Recognition Rate = 0.32 ).\\

Finally, looking specifically at the distance type, we noticed that on average the intersect distance was the best for each type of histogram.
The average is calculated taking into account the six test cases \(num_bins = 5, 10, 15, 20, 30, 50\).

\begin{figure*}[h!]
    \centering
    \includegraphics[height=2.3in]{img/best_dist.png}
     \caption{Best Distance}
\end{figure*}






\newpage
\section*{Report 3 - Performance Evaluation}

For this exercise, after implementing the \textbf{rpc\_module} functions, we plotted the RPC curves for different histogram types, distances and number of bins. After experimenting with the number of bins, we got different results regarding the distances. In the following picture we are going to see the plots for \textbf{RG histogram} with 10, 20 and 30 bins.\\ \\
\begin{figure}[h!]
     \centering
     \begin{subfigure}[b]{0.3\textwidth}
         \centering
         \includegraphics[width=\textwidth]{img/plots/RG_10.JPG}
         \caption{RG with 10 bins}
         \label{fig:y equals x}
     \end{subfigure}
     \hfill
     \begin{subfigure}[b]{0.3\textwidth}
         \centering
         \includegraphics[width=\textwidth]{img/plots/RG_20.JPG}
         \caption{RG with 20 bins}
         \label{fig:three sin x}
     \end{subfigure}
     \hfill
     \begin{subfigure}[b]{0.3\textwidth}
         \centering
         \includegraphics[width=\textwidth]{img/plots/RG_30.JPG}
         \caption{RG with 30 bins}
         \label{fig:five over x}
     \end{subfigure}
        \caption{RG histograms}
        \label{fig:three graphs}
\end{figure}
We notice that \textcolor{ForestGreen}{intersect} performs better than \textcolor{red}{chi2} and \textcolor{blue}{l2} in both of the cases for RG histogram.\\ 
Now let's take a look at the performace of the distances in \textbf{RGB histograms} with 10, 20 and 30 bins. We notice nearly the same result as the previous case: \textcolor{ForestGreen}{intersect} performs better than \textcolor{red}{chi2} and \textcolor{blue}{l2} in both of the cases for RGB histogram. We notice that \textbf{ RGB} is the best histogram model we can use. Based on the results of the exercise 2, we can confirm that the \textbf{RGB with 15 bins} and \textcolor{ForestGreen}{intersect} is the ideal model. 
\begin{figure}[h!]
     \centering
     \begin{subfigure}[b]{0.3\textwidth}
         \centering
         \includegraphics[width=\textwidth]{img/plots/RGB_10.JPG}
         \caption{RGB with 10 bins}
         \label{fig:y equals x}
     \end{subfigure}
     \hfill
     \begin{subfigure}[b]{0.3\textwidth}
         \centering
         \includegraphics[width=\textwidth]{img/plots/RGB_20.JPG}
         \caption{RGB with 20 bins}
         \label{fig:three sin x}
     \end{subfigure}
     \hfill
     \begin{subfigure}[b]{0.3\textwidth}
         \centering
         \includegraphics[width=\textwidth]{img/plots/RGB_30.JPG}
         \caption{RGB with 30 bins}
         \label{fig:five over x}
     \end{subfigure}
        \caption{RGB histograms}
        \label{fig:three graphs}
\end{figure} \\
The final histogram is the \textbf{dxdy histogram.} In this case we notice a slightly similar performance from all the measurements, but as well in this case the best performing distance is \textcolor{ForestGreen}{intersect} 
\begin{figure}[h!]
     \centering
     \begin{subfigure}[b]{0.3\textwidth}
         \centering
         \includegraphics[width=\textwidth]{img/plots/dxdy_10.JPG}
         \caption{DxDy with 10 bins}
         \label{fig:y equals x}
     \end{subfigure}
     \hfill
     \begin{subfigure}[b]{0.3\textwidth}
         \centering
         \includegraphics[width=\textwidth]{img/plots/dxdy_20.JPG}
         \caption{DxDy with 20 bins}
         \label{fig:three sin x}
     \end{subfigure}
     \hfill
     \begin{subfigure}[b]{0.3\textwidth}
         \centering
         \includegraphics[width=\textwidth]{img/plots/dxdy_30.JPG}
         \caption{DxDy with 30 bins}
         \label{fig:five over x}
     \end{subfigure}
        \caption{DxDy histograms}
        \label{fig:three graphs}
\end{figure} \\







%------------------------------------------------
\end{document}
